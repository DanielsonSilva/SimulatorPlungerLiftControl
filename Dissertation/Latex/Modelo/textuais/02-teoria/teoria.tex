%%
%% Capítulo 2: Regras gerais de estilo
%%

\mychapter{Teoria}
\label{Cap:Teoria}

A teoria (ou referencial teórico) deve descrever as técnicas e metodologias existentes e desenvolvimentos anteriores que sejam essenciais ao entendimento do trabalho, sempre citando referencial teórico para cada técnica abordada. Aqui, as ferramentas matemáticas, mesmo que já conhecidas, podem ser revisitadas com as devidas citações. Geralmente, estas informações são obtidas em livros, \textit{survey papers} ou \textit{seminar papers}. É importante lembrar que o trabalho relatado em uma tese ou dissertação, muito mais do que em um artigo, deve ser auto-contido e  reproduzível.

Além de instruções sobre a montagem da teoria, este capítulo também apresenta considerações de ordem geral sobre a organização que deve ser adotada no seu documento, tais como número de
páginas, margens e subdivisões.

\section{Dimensões}
\label{dimensões}

Não há um número mínimo ou máximo de páginas para propostas de tema,
dissertações ou teses. Entretanto, se o seu documento for muito menor
do que a média pode transmitir uma ideia de falta de conteúdo a
apresentar. Por outro lado, um documento muito grande corre o risco de
só conseguir a atenção total do leitor no seu início, fazendo com que
as partes mais importantes, que geralmente estão no final do
documento, não sejam devidamente consideradas. Apenas para servir como
parâmetro, estão indicados a seguir os limites usuais quanto ao número
de páginas\footnote{Uma folha corresponde a uma página em impressão em
face simples e a duas páginas em impressão em face dupla} dos
documentos do PPgEEC da UFRN, adotando as margens e os espaçamentos
definidos neste modelo:
\begin{itemize}
\item Proposta de tema para exame de qualificação de mestrado:
entre 20 e 40 páginas
\item Proposta de tema para exame de qualificação de doutorado:
entre 30 e 50 páginas
\item Dissertação de mestrado:
entre 50 e 100 páginas
\item Tese de doutorado:
entre 80 e 150 páginas
\end{itemize}

O tamanho padrão para a fonte é de 12pt.  Para facilitar a escrita de
comentários, sugestões e correções da banca, recomenda-se o espaçamento
1.5 entre as linhas do texto e a impressão em um único lado da folhas
para os seguintes documentos:
\begin{itemize}
\item Proposta de tema para exame de qualificação;
\item Versão inicial de dissertação de mestrado; e
\item Versão inicial de tese de doutorado.
\end{itemize}
Para as versões finais de teses e dissertações, onde se busca uma
melhor qualidade visual e tipográfica do texto, deve-se utilizar
espaçamento simples entre as linhas e a impressão nos dois lados da
página.

As margens devem seguir os valores adotados neste documento, que podem
ser verificados no arquivo \texttt{principal.tex}. É importante notar
que, na versão final de teses ou dissertações, recomenda-se a
impressão nos dois lados da página. Por esta razão, a margem direita
em páginas pares deve ter o mesmo valor que a margem esquerda em
páginas impares e vice-versa, para que a encadernação fique
correta. Também em razão da impressão em frente e verso, os capítulos
devem sempre começar em uma página de número ímpar, com a eventual
inclusão de uma página em branco. O \LaTeX\ se encarrega de fazer
automaticamente estes ajustes.

\section{Divisões do documento e referências cruzadas}
%\label{Sec:divisoes}

Documentos do porte de uma tese ou dissertação devem ser subdivididos
em capítulos. O capítulo deve conter uma introdução e um fecho.

A introdução do capítulo fornece ao leitor uma breve descrição do que
será tratado no capítulo e não forma uma seção: para exemplificar, a
introdução deste capítulo é o parágrafo que precede a primeira seção.

O fecho do capítulo apresenta comentários finais sobre o que foi
desenvolvido no capítulo e/ou faz uma ligação com o que será visto no
capítulo seguinte; normalmente é colocado em uma seção específica,
denominada ``Comentários Finais'', ``Conclusões'', ``Resultados'',
``Avaliação Final'' ou qualquer outra denominação que se adéque ao
texto.

Capítulos são divididos em seções. O número ideal de seções é
impossível de se precisar. Entretanto, um capítulo com uma única seção
provavelmente deve ser agregado ao capítulo anterior ou posterior. Um
capítulo com quinze seções provavelmente deve ser subdividido em dois
capítulos.

Capítulos, seções e subseções devem ser rotulados para que possam ser
referenciados em qualquer parte do texto.  Isto é feito com o comando
\verb|\label{}|, que deve ser colocado logo após (nunca antes) o
comando que criou a seção, capítulo, etc. O parâmetro do comando
\texttt{label} é o nome simbólico que será usado para se fazer
referência a esta entidade dentro do texto, com o comando
\verb|\ref{}|. O nome pode ser qualquer coisa, mas não pode conter
acentos, por exemplo. Neste documento nós utilizamos a convenção de
prefixar os rótulos dos capítulos com \texttt{Cap:}, das seções com
\texttt{Sec:}, das equações com \texttt{Eq:} e assim por diante, mas
esta convenção não é obrigatória. Veja a seguir um exemplo de
utilização das referências cruzadas:
% quotation é um ambiente para citações, que ficam "recuadas" em
% relação ao resto do texto
\begin{quotation}
\dots no capítulo~\ref{Cap:Introducao} apresentamos um modelo de
capítulo de tese.
\end{quotation}
Note que, no código fonte deste trecho de frase, o espaço entre a palavra
\texttt{capítulo} e o comando \verb|\ref{}| foi escrito com
um \texttt{\~{}} e não com um espaço normal. O \texttt{\~{}} é o
comando \LaTeX\ para criar um espaço onde não se pode mudar de linha,
pois ficaria estranho se o texto ``no capítulo'' estivesse no fim de
uma linha e o número \ref{Cap:Introducao} no início da outra linha.

Existe uma particularidade no código fonte do parágrafo anterior. Para
se escrever:
\begin{quotation}
\dots o comando \LaTeX\ para criar\dots
\end{quotation}
se colocou depois do comando \verb|\LaTeX| um espaço
precedido de uma contrabarra, ao invés de um espaço normal. Isto porque
espaços depois de comandos são ignorados pelo \LaTeX; com um espaço
% Na linha anterior não houve necessidade do espaço com contrabarra
% depois do comando \LaTeX, pois ele é seguido por um ; e não um espaço
normal as palavras ficariam ligadas:
\begin{quotation}
\dots o comando \LaTeX para criar\dots
\end{quotation}
Ao invés do espaço precedido pela contrabarra, poder-se-ia também
utilizar um \texttt{\~{}}. A diferença é que neste caso o \LaTeX\ não
poderia fazer uma quebra de linha entre as palavras.

\section{Seções}
%\label{Sec:secoes}  %% labels não devem conter caracteres acentuados

Seções são divisões do conteúdo do capítulo. Esta divisão
deve ser lógica (temática) e não física (por tamanho).
Por exemplo, um capítulo que trata de \textit{software}
de sistema teria seções que tratam de montadores, ligadores,
carregadores, compiladores e sistemas operacionais.

Tal como capítulos, seções devem ser rotuladas para referência
em outras partes do texto. Seções são divididas em subseções.

\subsection{Subseções}
%\label{Sec:subsecoes}

Subseções são divisões de seções. No exemplo do texto sobre
\textit{software} de sistema, a seção referente a sistema operacional
conteria, por exemplo, subseções que tratam de arquivos, processos,
memória e entrada/saída.  Tal como seções, subseções são divisões
temáticas do texto.

\subsubsection*{Subsubseções}
%\label{Sec:subsubsecoes}

Subsubseções são divisões de subseções e não devem ser numeradas no
texto. O \texttt{*} após o comando \texttt{subsubsection*} instrui
o \LaTeX\ a não numerar a subsubseção. Esta mesma regra se aplica
a outros comandos. Por exemplo, \verb|\chapter{}|
inicia um capítulo, enquanto \verb|\chapter*{}|
inicia um capítulo sem número. O comando \texttt{chapter*} foi
usado no arquivo \texttt{resumos.tex} para criar os capítulos não
numerados referentes ao resumo e ao \textit{abstract}.

\section{Índices}

O \LaTeX\ é capaz de gerar automaticamente o índice do texto
(sumário), os índices de figuras e de tabelas e uma lista de símbolos
ou glossário.

\subsection{Sumário}
\label{Sec:sumario}

Todas as divisões numeradas (capítulos, seções e subseções) são
automaticamente incluídas no sumário. Ao se criar uma nova divisão é
necessário compilar duas vezes o texto com o \LaTeX: na primeira
compilação será percebida a inclusão da nova divisão, enquanto na
segunda será gerado o índice atualizado. Esta mesma necessidade de uma
dupla compilação aparece quando se acrescenta qualquer nova referência
cruzada: uma nova figura ou tabela, uma nova referência bibliográfica,
etc.

Além das divisões que são incluídas automaticamente no sumário, pode-se
incluir manualmente outras informações. Os índices, por exemplo, não
são incluídos automaticamente no sumário. Verifique no arquivo
\texttt{principal.tex} o que deve ser feito para fazer esta inclusão.

\subsection{Listas de figuras e tabelas}
\label{Sec:listasfigtab}

Estas listas são geradas automaticamente a partir dos
\texttt{caption}'s de todos os ambientes \texttt{figure} e
\texttt{table}. Maiores detalhes sobre estes ambientes serão apresentados
no capítulo~\ref{Cap:Problema}.

\subsection{Lista de símbolos (glossário)}
\label{Sec:glossario}

Este ambiente pode ser utilizado para produzir uma lista de símbolos,
um glossário ou uma lista de abreviaturas. Ao utilizar pela primeira
vez uma entidade que precise de definição, o autor, ao final do
parágrafo, gera a entrada para o glossário. A título de exemplo,
foram incluídos na lista alguns símbolos e abreviaturas que aparecem
no texto a seguir:

\begin{quotation}
As primeiras teses apresentadas no PPgEEC da UFRN eram datilografadas
manualmente. Para escrever uma fórmula simples como:%
\nomenclature[aa]{UFRN:}{Universidade Federal do Rio Grande do Norte}%
\nomenclature[az]{PPgEEC:}{Programa de Pós-Graduação em Engenharia Elétrica}%
\begin{equation}
\omega = \omega_0 + \alpha \cdot \Delta t
\label{Eq:glossario}
\end{equation}%
% Há problemas quando se põe uma quebra de linha em um comando \nomenclature.
% Para evitá-lo, comenta-se com um % todos os fins de linha das linhas que
% contêm o comando e da linha imediatamente anterior.
\nomenclature{$\omega$}{velocidade angular do corpo}%
\nomenclature{$\omega_0$}{velocidade angular inicial do corpo}%
\nomenclature{$\alpha$}{aceleração angular do corpo}%
\nomenclature{$\Delta t$}{tempo decorrido desde o instante em que foi medida
a velocidade inicial até o instante presente}%
os autores tinham que desenhar os símbolos manualmente ou, para os
mais afortunados, trocar inúmeras vezes a esfera da sua máquina de
datilografia.
\end{quotation}

A ordem de aparição dos verbetes no glossário é a seguinte:
inicialmente os símbolos, depois os números e por último as
\textit{strings}. Você pode modificar esta ordem incluindo um
parâmetro adicional no comando \verb|\nomenclature[opcional]{simb}{signif}|. No exemplo, este
parâmetro foi utilizado para colocar UFRN antes de PPgEEC.

\section{Bibliografia}
\label{Sec:citacoes}

As referências bibliográficas são incluídas dentro de um ambiente
específico para este fim, através dos comandos
\verb|\begin{thebibliography}| e \verb|\end{thebibliography}|. Um
documento só pode conter um único destes ambientes. A referência aos
documentos no texto é feita usando-se referências cruzadas e chaves
simbólicas, da mesma forma que as equações e as figuras. Cada
documento dentro de um ambiente \texttt{thebibliography} é introduzido
por um comando \verb|\bibitem|. O argumento obrigatório (entre chaves)
do comando \texttt{bibitem} é a chave simbólica pela qual o documento
será citado no texto, usando o comando \verb|\cite|.  O argumento
opcional (entre colchetes) do comando \texttt{bibitem} é a expressão
que será inserida tanto no texto, no local onde a referência foi
citada, quanto na lista de referências bibliográficas, como etiqueta
do documento em questão.

O ambiente \texttt{thebibliography} pode ser digitado diretamente pelo
usuário ou gerado automaticamente a partir de um arquivo de
informações bibliográficas. A digitação manual tem a vantagem de
tornar o documento autocontido, enquanto a geração automática permite
um melhor reaproveitamento das informações e uma maior uniformidade
das referências nos diversos documentos. Sempre que possível,
aconselha-se a geração automática, que é feita pelo aplicativo
\BibTeX.

A principal vantagem da geração automática de bibliografias é que se
pode manter um arquivo único com todas as referências bibliográficas
que foram ou podem vir a ser usadas em algum dos seus documentos
(artigos, tese, etc.).  O \BibTeX\ se encarrega de verificar quais
delas foram efetivamente citadas no documento sendo processado e gerar
um ambiente \texttt{thebibliography} que contém apenas os documentos
necessários.

As informações bibliográficas devem ser salvas em um arquivo no
formato \BibTeX\ e com extensão \texttt{.bib}. O formato \BibTeX\ 
permite referenciar diferentes tipos de documentos:
\begin{itemize}
\item artigos em revistas~\cite{art-solimaes03};
\item artigos em anais de simpósios~\cite{proc-gates01};
\item artigos em coletâneas de artigos~\cite{col-pinto00};
\item capítulos de livros~\cite{inbook-santos00};
\item anais de simpósios~\cite{proc-sbrc02};
\item livros~\cite{book-lamport94};
\item teses de doutorado~\cite{phd-gates01};
\item teses de mestrado~\cite{msc-santos03};
\item relatórios técnicos~\cite{rep-omg2000};
\item manuais técnicos~\cite{man-orbix99};
\item trabalhos não publicados~\cite{unp-sichman02};
\item páginas na Internet~\cite{hp-novet03} (a data é o dia do
      último acesso à página);
\item miscelânea~\cite{misc-cruz03}.
\end{itemize}

O arquivo \BibTeX\ não contém nenhuma informação de formatação. Esta
formatação é definida através do comando \verb|\bibliographystyle|.
Este modelo inclui um arquivo de estilo (\texttt{ppgee.bst}) que gera
as referências no padrão adotado para os documentos do PPgEEC. Este
estilo é baseado no padrão \texttt{jmr} do pacote \texttt{harvard}, com
as modificações necessárias para a língua portuguesa. Para maiores
informações, veja a documentação do pacote \texttt{harvard}, disponível
na Internet ou na maiorias das instalações \LaTeX.

A lista de referências bibliográficas é gerada pelo comando
\verb|\bibliography|, cujo parâmetro obrigatório é o nome do(s) arquivo(s)
que contém(êm) as informações bibliográficas.  O \BibTeX\ se encarrega
de extrair deste(s) arquivo(s) as referências citadas, formatá-las de acordo
com o estilo escolhido e gerar o ambiente \texttt{thebibliography}
correspondente. Este ambiente é salvo em um arquivo com terminação
\texttt{.bbl}, que é automaticamente inserido no documento no local do
comando \verb|\bibliography|. Este procedimento pode ser melhor
compreendido analisando-se os arquivos \texttt{principal.tex} e
\texttt{bibliografia.bib}, além do arquivo \texttt{principal.bbl}
gerado automaticamente pelo \BibTeX.

Uma recomendação importante é que as citações não fazem parte do
texto; portanto, as frases devem fazer sentido mesmo que as expressões
de citação sejam removidas. Para exemplificar, não se deve usar:
\begin{quotation}
\dots conforme demonstrado por \cite{art-solimaes03}.
\end{quotation}
e sim:
\begin{quotation}
\dots conforme demonstrado por \citeasnoun{art-solimaes03} (para tal citação, utiliza-se o comando \verb|\citeasnoun|).

\dots conforme demonstrado na literatura
\cite[e.g. ]{art-solimaes03}.\footnote{A abreviatura e.g.
significa por exemplo. Vem do latim \emph{exempli gratia}. Também se
usa, para o mesmo caso, v.g. (\emph{verbi gratia}) ou simplesmente
p.ex.}
\end{quotation}

\section{Considerações finais}

O simples fatos de utilizar corretamente uma boa ferramenta de
formatação de textos e de seguir as recomendações quanto ao estilo
não garantem a qualidade do documento produzido. O principal
aspecto a ser levado em conta é a qualidade da redação.

Propostas de tema, dissertações e teses devem ser escritas em
linguagem técnica, que difere em alguns aspectos da linguagem
literária.  Em textos das ciências exatas e tecnológicas, o objetivo
principal é a clareza e a correção, algumas vezes com um certo
prejuízo da estética literária. Algumas recomendações quanto à
redação são as seguintes:
\begin{itemize}
\item Evite períodos longos, reduzindo o número de apostos, orações
subordinadas, pronomes relativos (o qual, que, cujo, o mesmo, do qual,
etc.) e inversões na ordem normal de aparecimento dos elementos da
frase (sujeito, verbo, predicado). Divida períodos longos em várias
frases menores, mesmo que seja necessário repetir alguns termos. Por
exemplo:
\begin{quotation}
\emph{Esta versão foi concebida por ocasião da disciplina de Sistemas
de Transmissão de Dados, que tinha como principal objetivo, fazer uso
de uma plataforma móvel onde pudesse ser aplicada uma técnica de
transmissão de dados, possibilitando dessa forma o controle desta.}
\end{quotation}
pode ser substituído por:
\begin{quotation}
\emph{Esta versão foi concebida durante a disciplina de Sistemas de
Transmissão de Dados. O principal objetivo desta primeira versão
era aplicar uma técnica de transmissão de dados a uma plataforma
móvel. O uso da técnica de transmissão de dados tornou possível
o controle da plataforma móvel.}
\end{quotation}
\item Verifique a pontuação empregada para não separar por vírgulas
o sujeito do verbo, como no exemplo anterior em ``\dots\emph{tinha
como principal objetivo, fazer uso de}\dots'', nem trocar pontos por
vírgulas ou vice-versa. Estes erros geralmente aparecem associados a
períodos muito longos, como nos dois exemplos a seguir:
\begin{quotation}
\emph{Quando a operação é realizada sem o sistema de recepção, o
veículo se desloca e realiza tarefas, baseadas na programação do
dispositivo microcontrolador e leitura de sensores, como a odometria,
por exemplo, a tabela abaixo lista a função atribuída a cada pino.}
\end{quotation}
\begin{quotation}
\emph{A garra é dotada também de um conjunto de sensores mecânicos
de fim de curso tipo contato seco normalmente aberto (NA), nas
posições de máxima elevação e abertura, bem como dois sensores desse
mesmo tipo nas pinças da garra que respondem a uma pequena pressão em
sua superfície, sendo responsáveis por pressionar o elemento objeto
que se deseja recolher. Evitando que o mesmo escorregue quando
erguido.}
\end{quotation}
\item Evite o uso de adjetivos que expressam julgamentos de valor de
maneira não quantificável. Por exemplo, ao invés de dizer que os
resultados foram excelentes, diga que em 98\% dos experimentos o erro
foi menor que 1\%, deixando ao leitor a tarefa de concluir se estes
resultados são excelentes ou não.
\end{itemize}

No que diz respeito aos recursos de formatação, lembre-se que o
\LaTeX\ já faz a maior parte do trabalho para você, seguindo padrões
que foram definidos por especialistas para garantir uma boa qualidade
tipográfica. Portanto, tente não modificar ``manualmente'' a
formatação gerada. Para isto, evite sempre que possível os seguintes
recursos, pois um texto ``limpo'' é mais agradável de ler que um texto
``enfeitado'':
\begin{itemize}
\item letras maiúsculas em palavras onde elas não têm justificativa
gramatical: nomes de meses (janeiro e não Janeiro), apenas para dar
ênfase (\dots foi desenvolvido um Robô Móvel com rodas\dots), etc.
\item \textbf{o uso indiscriminado de negrito;}
\item \textit{o uso de itálico, exceto em palavras em outra língua
ou em definições de termos que aparecem pela primeira vez;}
\item texto em fontes diferentes: \texttt{espaçamento uniforme}, \textsf{sem
serifa}, \textsc{caixa alta}, etc.
\item \underline{o uso de texto sublinhado;}
\item o uso excessivo de notas de rodapé\footnote{Notas de rodapé
são estas expressões no pé da página.}.
\item palavras em língua estrangeira quando existe uma equivalente
em português (desempenho e não \textit{performance}) ou neologismos
não justificados (downloadar, randômico, etc.)
\end{itemize}
